\section{The Code}
\subsubsection{1/5 to 13/5}

\subsection{Added functionalities}

The code is being builded over Matthias Gallé PhD thesis code \cite{GThesis}. 
Many things should be refactored to obtain a clean code and some 
functionalities for l,k substatibility added, listed below:
\begin{itemize}
	\item u-y-v yield class
	\item Cartesian Product within all yields to obtain all possible crosses
	\item Greedy algorithm to choose a combination u-v in the sequence
	\item u-y-v yields replacement by Non-Terminals 
	\item minor Tests
\end{itemize} 

All these things were already added, but the code become a monster. It doesn't 
follow any convention at all, nor a design. Global variables are everywhere and
classes don't have encapsulation at all...

So I decided to remake a code for the next generation working on this topic. Of
course it's based on \cite{GThesis}, but it's coded with a goal in mind and following
some C++ conventions and commented so it's easier to follow and understand what
is happening.

Also, each function is being commented and an html Documentation created 
automatically with Dioxygen (see \emph{html} or \emph{latex} folder in the project root).

\subsubsection{10/6}

\subsection{First implementation}

In the first implementation we use to select our u-y-v yields a score function 
similar to that showed in section \textbf{1.1.1}. 
We use this formula to obtain the u-y-v that will reduce the most our grammar in the 
each step. 

\subsubsection{Yield replacement}

Once the yield to be replaced is chosen, we replace it in out grammar in the following way:
\begin{itemize}
	\item Search for Maximal Repeats in the contatenation of productions:
	\begin{center}
	$P_1 \$ P_2 \$ ... \$ P_n \$$
	\end{center}
	A maximal repeat couldn't have a part in a production $P_i$ and the other in 
	a production $P_j$ with $j \neq i$, so that must be checked.
	\item Once all Maximal Repeats are founded we crossed them, creating all 
	u-y-v possibilities.
	\item each u-y-v should be applied to the score function and from it we select
	the one that minimize the size of our grammar.
	\item the yield chosen is replaced in the production concatenation and from it
 	the new grammar is retrieved.

\end{itemize}

\subsection{Compression}

Until now we were talking just about grammar size, but now we have to start with
the compression. We got to find the right score function to obtain the better
compression rate for the sequence.