
	\chapter{New approach proposed}

	Adding rules with a certain type of ambiguity allows us to take more constituents to be replaced, 
	because we aren't limitated only by the repeats. \\

	That's why we are going to introduce
	a kind of CFL (Context-Free languages) that can be used for that purpose: the \emph{k,l substitutable}.\\

	\section{Local-context K,L Substitutable Language}

	We will denote by \textbf{yield} of N ($yields(N)$) the strings that the non terminal
	N yields. \\
	$\bm{u}$ and $\bm{v}$ are, respectively, a prefix and a suffix of a yield and the will be called the \textbf{contexts} of the yield. 

	In an informal way, we can define the local-context k,l substitutable languages, 
	as a subset of the CFL, where we have yields formed
	by contexts $u$ and $v$ of size k and l, sourrounding different insides $y_i$ that can be 
	interchangeable within them.

	A language $L$ is, $k$,$l$-local context substitutable if for any 
	$x_1,y_1,z_1,x_2,y_2,z_2,x_3,z_3 \in \sum^*$ and $u, v \in \sum^{k,l}$, 
	such that \\
	 $uy_1v, uy_2v \neq \lambda, \\
	 x_1uy_1vz_1 \in L$ $\wedge\ x_3uy_2vz_3 \in L$	$\Rightarrow (x_2uy_1vz_2 \in L \Leftrightarrow x_2uy_2vz_2 \in L)$.

	Definition could be find in \cite[p. 3-4]{KL}. \\

	Let us clarify with an example: \\

	With contexts we mean $u$ and $v$ subsequences of our grammar.
	For example if we had:\\
	\begin{center}
	$S \rightarrow agtggtgaaccaagatggaccacacggttggaccg$
	\end{center}

	Two possible contexts could be, for example, $u = ag$ and $v = gt$.

	\begin{center}
	$S \rightarrow \textbf{u}tg\textbf{v}gaacca\textbf{u}atggaccacacg\textbf{v}tggaccg$
	\end{center}

	and by pairing them in an ordered way we can obtain different inside subsequences surrounded
	by those given contexts. In this case, those insides are: \\
	\begin{center}
	$y_1 \rightarrow tg$ \\
	$y_2 \rightarrow atggaccacacg$
	\end{center}

	The insides will give us the ambiguity: \\

	\begin{center}
	$S \rightarrow \textbf{N}gaacca\textbf{N}tggaccg$\\
	$N \rightarrow u I v$ \\
	$I \rightarrow y_1 | y_2$
	\end{center}

	The problem here, is that if we want to decode the grammar and go back to our
	first sequence, there is no way to know which Inside to use in each replacement, as
	we did with the SLG approach. 

	In following sections we will see how to resolve it.