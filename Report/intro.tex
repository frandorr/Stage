\chapter{Introduction}

The main idea of this work is to try to compress DNA sequences by means of
grammar inference. The grammars used here is a subset of the context free grammars,
called \emph{k,l-local context substitutable} \cite{Gaelle}. \\

An example of these grammars:\\
Lets say we have the follow sequence: \\
\begin{center}
agtggaccagtggaaccagtggaaccagtgacgt
\end{center}

Lets take two contexts u and v with lengths $k=2$ and $l=3$: \\
\begin{center}
$u = gg$\\
$v = cca$
\end{center}

In our sequence:\\
agt\textbf{gg}a\textbf{cca}gt\textbf{gg}aa\textbf{cca}gt\textbf{gg}aa\textbf{cca}gtgacgt \\

After we identify the u and v in the sequence we should group them in pairs and
see what substrings are within them.
\begin{itemize}
	\item \textbf{gg}a\textbf{cca} inside: a
	\item \textbf{gg}aa\textbf{cca} inside: aa
	\item \textbf{gg}aa\textbf{cca} inside: aa
\end{itemize}

We can replace these yields with a non terminal N and the insides with a non terminal I 
in the follow way:\\

Original sequence:\\
agt\textbf{ggacca}gt\textbf{ggcca}gt\textbf{ggcca}gtgacgt \\

Yields replaced by non terminal: \\
S $\rightarrow$ agt\textbf{N}gt\textbf{N}gt\textbf{N}gtgacgt \\
N $\rightarrow$ gg I cca\\
I $\rightarrow$ a $|$ aa\\

Hence, we have found a k,l-local context substitutable grammar for representing
our sequence in a compressed way.