% Preview source code from paragraph 5 to 51

\chapter{Brief Daily Report}

\section{Formulas (only in size of characters)}

Based on MDL principle I want to find an hipothesis that minimize:
\begin{center}
$|H| + |S/H|$
\end{center} 

In our case, that would be a grammar:
\begin{center}
$|G| + |S/G|$
\end{center} 


Naming $G_{0}$ the original Sequence:
\begin{center}
 $G_{0}: S \rightarrow -uy_{1}v ... uy_{n}v-$
\end{center}

We want to find $G_{1}$ so: $|G_{1}|+|S/G_{1}|<|G_{0}|$.\\

Or in general we want: $|G_{n}|+|S/G_{n}|<|G_{n-1}|$ 


\subsection{What we gain?}

\subsubsection{Revised 18/4}

Everytime we replace a yield we want to gain something, if we lost it's better to leave the grammar as it is.
If we have:
\begin{itemize}
	\item $|u| = k, |v| = l$
	\item $p$ yields ($uy_iv$ with $i=0..q \leq p$) to be replace
\end{itemize}

We could have two cases:
\begin{enumerate}
	\item $q=p \rightarrow$ worst case.
	\item $q<p$	
\end{enumerate}

\begin{itemize}
	\item $\bm{q=p}$: \\
		$|G_1| = |G_0| -p*k -p*l -\sum_{i=1}^{p}|y_i| + p + 1 + k + l + 1 + 1 + \sum_{i=1}^{p}|y_i| + (p-1)$. \\
		
		That means:\\

		\begin{itemize}
			\item $|G_1|$: size of next grammar 
			\item $|G_0|$: size actual grammar
			\item $-p*k-p*l$: remove of contexts u-v
			\item $-\sum_{i=1}^{p}|y_i|$: remove of insides $y_i$ 
			\item $+ p$: yields replaced by N symbol
			\item $+1$: N production symbol
			\item $+k+l$: u and v in N production
			\item $+1+1$: I production symbol + I symbol in N production
			\item $+ \sum_{i=1}^{p}|y_i|$: $y_i$ in I production
			\item $+(p-1)$: pipes in I production
		\end{itemize}


		and we want this whole thing after $|G_0|$ to be below 0. We got in a few trivial steps:
		$k+l+2p+2<kp+lp$\\
		According to this formula we obtain the following result:\\

		\begin{itemize}
					\item 	$l>2-k$ 
					\item   $p>\dfrac{k+l+2}{k+l-2}$
		\end{itemize}	

		For example, setting $|u|=3$ and $|v|=5$ we need:

		\begin{center}
			$p> \lfloor \dfrac{3+5+2}{3+5-2} \rfloor = \lfloor \dfrac{10}{6} \rfloor = 1 $
		\end{center}
		That means that having more than 1 yield to replace (doesn't matter the size of the inner $y_i$) will give us a gain.

		This bound is really useful because when we test some contexts we can calculate if the replacement is going to reduce or not our grammar.

		Lets try another example:

		setting $|u|=1$ and $|v|=2$ we need:
		\begin{center}
			$p> \lfloor \dfrac{1+2+2}{1+2-2} \rfloor = \lfloor \dfrac{5}{1} \rfloor = 5 $
		\end{center}

	\item $\bm{q<p}$:\\
		This case is more interesting in terms of grammar size reduction, because besides reducing contexts we gain also with the inner of the yields.

		The formula is similar but we have to substract the inner sizes that occurs more than one time:
		\begin{center}
		$|G_1| = |G_0| -p*k -p*l -\bm{\sum_{i=1}^{q}|y_i|(\o(y_i)-1)}+ p + 1 + k + l + 1 + 1 + (\bm{q}-1)$. \\
		\end{center}

		where $q$ is the number of different inners and $\o(y)$ is the occurrences number of y. 

		Again, doing some trivial steps if we want to reduce the size, the follow must state: \\

		$p(-k-l+1) + 2 +k+l+q < \sum_{i=1}^{q}|y_i|(\o(y_i)-1)$\\

		For example, setting $|u|=2$, $|v|=1$, $p=3$, $q=2$ we have:
		\begin{center}
		$3(-2-1+1) + 2 +2+1+2 < \sum_{i=1}^{2}|y_i|(\o(y_i)-1)$\\
		$1<|y_r|$
		\end{center}
		where $y_r$ is the $y$ with 2 occurrences. 
		So if $|y_r| =1$ we are not gaining, but if it is bigger we gain in the grammar size where in the case of $\bm{q=p}$ we didn't (with $k=1,l=2$ we needed $p>5$ and now just $p=3$)
	

\end{itemize}

% At last, we want to formalize this results and look at the gain in number of letters and not in $l,k$ and $y_r$.
% For example: if I replace $n$ letters and $m$ of them are repetead, what is my gain? Gotta think about it next week!

% \subsubsection{22/4}

% Instead of measuring the contexts size and the inners size, we could measure how many letters and their occurrences are we replacing.



% Comparing the new hipothesis with the old one we can analyse our worst
% scenario with such formula. This scenario is when the number of occurrences of each $y_i$ is 1, so we'd have $k=n$. 

% Fixing the size of $u$ and $v$ in for example 3 we have that:

% $|G_{0}|-(n-1)6+(n-1)+n+2<|G_{0}|$

% $-6n+6+2n+1<0$

% $-4n<-7$

% $n>7/4$ 

% We can see that altough we have no repeats inside the contexts we
% can gain in the size just by means of $u$and $v$



