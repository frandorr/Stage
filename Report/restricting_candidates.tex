\chapter{Restricting set of context candidates}

	There are many ways to choose different contexts in our sequence. To test all of them in each iteration is kind of expensive. Because of that,
	we should find an heuristic, that gives us good results by searching in a smaller
	subset of all possible contexts.

	\section{Crossing Maximal Repeats}

	When replacing simple exact words, a good way is choosing them within the
	sequence Maximal Repeats (MR) as stated in \cite{GThesis}, because it is fast
	using a suffix tree data structure.

	We start thinking in a similar approach, by crossing all MR
	and obtaining contexts u-v from them ($u = MR_i$, $v = MR_j$).

	This approach could only be used in small datasets, because it's highly demandant
	in the resources it uses, because we should cross them all, with a complexity
	of $\O(|MR|^2)$ that gets easily really big.


	\section{Contexts as subsequence of MR}

	The idea is to obtain the contexts u-v from the very same maximal repeat. As replacing
	words from MR gave great results in \cite{GThesis}, if we get the u-v from
	them the intution says that we should get similar results. The computation time and
	resources used improve, because we are just crossing prefix and suffix of each
	MR and this is less costly than crossing all MR.
	The counterpart is that we are cutting some possibilities.

	\section{Contexts from best MR}

	Here we first select the best MR as the SLG problem in \cite{GThesis} and we obtain
	from it the possible contexts.
	It's a really fast approach because we are using only one MR, but we don't know
	if the best MR word will guide us to best contexts replacements.

	\section{Limitations on U, V sizes}

	If we are using Maximal Repeats, it is usually common in the experimental results, that the sizes of u and v based on them	don't go over a certain small limit $s$, so we can fix the search to maximal repeats
	of size $s$ increasing the performance and obtaining similar results. \\

	We think, this happens because the possibility of pairing two small contexts is greater
	than pairing big contexts.
